% !Mode:: "TeX:UTF-8"
\appendix{致\quad 谢}

晋太元中,武陵人捕鱼为业,缘溪行,忘路之远近。忽逢桃花林,夹岸数百步,中无杂树,芳草鲜美,落英缤纷,渔人甚异之。复前行,欲穷其林。林尽水源,便得一山,山有小口,仿佛若有光,便舍船从口入。初极狭,才通人,复行数十步,豁然开朗。土地平旷,屋舍俨然,有良田美池桑竹之属。阡陌交通,鸡犬相闻。其中往来种作,男女衣著,悉如外人。黄发垂髫,并怡然自乐。
  见渔人,乃大惊,问所从来。具答之。便要还家,设酒杀鸡作食。村中闻有此人,咸来问讯。自云先世避秦时乱,率妻子邑人来此绝境,不复出焉,遂与外人间隔。问今是何世,乃不知有汉,无论魏晋。此人一一为具言所闻,皆叹惋。余人各复延至其家,皆出酒食。停数日,辞去。
  此中人语云,不足为外人道也。既出,得其船,便扶向路,处处志之。及郡下,诣太守说如此。太守即遣人随其往,寻向所志,遂迷不复得路。南阳刘子骥,高尚士也,闻之,欣然规往,未果。寻病终。后遂无问津者。

\vspace{9pt}%
\begin{minipage}[t]{0.8\columnwidth}%
\begin{flushright}
作者姓名
\par\end{flushright}

\begin{flushright}
\today
\par\end{flushright}%
\end{minipage}
