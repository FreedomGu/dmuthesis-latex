\appendix{攻读博士学位期间取得的研究成果}

这部分可以借鉴下面方法做成一个表格,也可以直接用\verb|enumerate|等格式罗列

一、已发表(包括已接受待发表)的论文,以及已投稿、或已成文打算投稿、或拟成文投稿的论文情况(只填写与学位论文内容相关的部分):


% \begin{table}
% \centering

\begin{longtable}{|>{\centering}m{0.5cm}|m{2.3cm}|>{\centering}m{3.5cm}|m{2.6cm}|m{2.0cm}|m{1.3cm}|m{0.9cm}|}
 \hline
序号 & 作者(全体作者,按顺序排列) & 题\hspace{1em}目 & 发表或投稿刊物名称、级别 & 发表的卷期年月、页码 & 相当于学位论文的哪一部分(章、节) & 被索引收录情况\tabularnewline
 \hline
\endfirsthead
\multicolumn{3}{l}{\small 续上表}\\
 \hline
序号 & 作者(全体作者,按顺序排列) & 题\hspace{1em}目 & 发表或投稿刊物名称、级别 & 发表的卷期年月、页码 & 相当于学位论文的哪一部分(章、节) & 被索引收录情况\tabularnewline
 \hline
\endhead
\endfoot
 \hline
\endlastfoot
1 & 全部作者 & 题目
         & 发表刊物 & 发表时间 & 第\ref{ch:intr}章 & SCI\tabularnewline
\hline
2 & 全部作者 & 题目
        & 发表刊物 & 发表时间 & 第\ref{ch:intr}章 & SCI\tabularnewline
\hline
3 & 全部作者 & 题目
         & 发表刊物 & 发表时间 & 第\ref{ch:intr}章 & SCI\tabularnewline
\hline
4 & 全部作者 & 题目
         & 发表刊物 & 发表时间 & 第\ref{ch:intr}章 & SCI\tabularnewline
\hline
5 & 全部作者 & 题目
         & 发表刊物 & 发表时间 & 第\ref{ch:intr}章 & SCI\tabularnewline
\hline
6 & 全部作者 & 题目
         & 发表刊物 & 发表时间 & 第\ref{ch:intr}章 & SCI\tabularnewline
\hline
7 & 全部作者 & 题目
         & 发表刊物 & 发表时间 & 第\ref{ch:intr}章 & SCI\tabularnewline
\hline
8 & 全部作者 & 题目
         & 发表刊物 & 发表时间 & 第\ref{ch:intr}章 & SCI\tabularnewline
\hline
9 &  &  &  &  &  & \tabularnewline
\hline
10 &  &  &  &  &  & \tabularnewline
\hline
11 &  &  &  &  &  & \tabularnewline
\hline
12 &  &  &  &  &  & \tabularnewline
\hline
13 &  &  &  &  &  & \tabularnewline
\hline
14 &  &  &  &  &  & \tabularnewline
\hline
15 &  &  &  &  &  & \tabularnewline
\hline
16 &  &  &  &  &  & \tabularnewline
\hline
17 &  &  &  &  &  & \tabularnewline
\hline
18 &  &  &  &  &  & \tabularnewline
\hline
19 &  &  &  &  &  & \tabularnewline
\hline
20 &  &  &  &  &  & \tabularnewline
\hline
\end{longtable}
% \end{table}

\newpage
二、与学位内容相关的其他成果(包括专利、著作、获奖项目等)

\begin{enumerate}[label=\arabic*,itemsep=5pt,topsep=10pt]
\item 专利 
\item 著作 
\end{enumerate}

