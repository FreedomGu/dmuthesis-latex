% !Mode:: "TeX:UTF-8"
\appendix{附\quad 录}

\section{定理证明} %附录1

联系,不再是简单的代码“连接”,而是两个不同关系人对于同一网页的共同“喜爱”。自媒体不再是Web2.0的唯一模式,150的巴顿数和六度关系理论,在社交网络架构下得到了充分释放。人的要素,前所未有地被放大了,这种放大,对应着2000年以前雅虎时代的网络管理员至上,对应着2005年Google横扫的机器算法至上。尽管Facebook并非无懈可击,每个人的所有真实想法,并不可能完全被移植记录到社交网络之上,并且过度的信息透明,彻底模糊了个人空间与公共空间之间的物理界限。但是社交网络依旧充满活力,这种活力来自于先天性的降低信息噪音,来自于毫不设防的新一代人生观与现实世界的有限隐私认同,来自新社交关系下的不对称权利关系重写。
  该是一笔好买卖——有什么能比承诺一个热衷于编程和网络的电脑呆子拥有令人艳羡的哈佛高端社交生活,有机会约到那些从不涉足计算机实验室的女孩子更具备诱惑性?然而扎克伯格的野心显然不仅于此,Facemash的成功,使他坚信,这是一场网络社交革命的前奏:既然对于在网络上浏览身边同学的信息并与其他人分享能够令整个校园如痴如狂,那么建立一个能够包含更多可见信息,同时可以供用户分享、查看的网站无疑是一种更令人激动的服务——与Myspace、Friendfinder等传统社交网站不同的是,用户必须以真实身份注册,在发出的邀请得到对方确认后才能与之联络并共享一切——它将是你现实中人际关系图的网络映射。
  无论如何,在接下来的两个月里,扎克伯格和温克沃斯兄弟、纳伦德拉互发了52封电
 

\section{源代码} %附录2


家罗宾·顿巴(Robin Dunbar)教授在1992年研究计算认为,人类大脑的逻辑和记忆力结构,注定了大脑可以容纳148人的稳定社交关系,按照这个巴顿数的规律,一个社会群组合适的规模大致就是150人,超过这个数字就无法有效地沟通和协作。恰恰社交网络的流行,迎合了巴顿数的生理本能需求,互联网上的交流界面和使用体验,无形中在参照着巴顿数为依据构建社交结构组织,从而规避了传统互联网海量信息筛选机制下的个人信息过载。实际上,我们完全可以将Facebook的崛起视作一种意义。”安东尼奥·瓦格斯在《纽约客》如此评述Facebook的全球效益。实际上,在2010年底,一名叫保罗·巴The Secret History of Social Networking)的系列节目。主持人走访了全球各地各路社交网络的幕后英雄,将这股炙手可热的潮流的起源追溯到近40年前,认为社交网络的理念本质上来源于上世纪六七十年代美国的反文化运动思潮,是当年嬉皮士和第一批网络黑客们关于平等、博爱的乌托邦实验的自然延伸。
  Friendster(建于2002年)
  其实Facebook并不是社交网络实名注册制的急先锋,Friendster才是。
  如今人们都在欢呼Facebook采用真实身份注册这一行为的革新性,然而,当年首次