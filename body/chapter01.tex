\chapter{为什么要做这个模板}\label{ch:intr}

因为使用Microsoft 
Word写学术论文是一件非常自我摧残的事情,要花费大量的时间去学习Word排版,
弄不好还要花费别人大量的时间帮忙排自己的论文,还要搭上人情,请人家吃饭什么的,
在别人心中留下一个小小的印记:“切,连排个版都不会!”

{\xiaoerhao 关键是用Word写学术论文本来就是一件十分{\color{red}不}靠谱的事情,而且完全有更简单,更专业的解决方法 —— {\color{blue}\LaTeX} !}

既然你下载了这个模板,而且有看文档的好习惯,那么恭喜你吞下了那颗红色小药丸!

Welcome to the real world!

其实下面的内容你都没有必要看的,就是例行的啰嗦啰嗦,你可以直接跳到
第\ref{ch:install}章。(看到了吧,pdf文档里面可以有超链接,你Word弄个超链接
出来费劲死了。)

\section{模板说明}
\label{sec:fastguide}

\subsection{模板特性}
\label{sec:features}

这个模板是大连海事大学硕博毕业论文\LaTeX{}模版,中文解决方案是~Xe\TeX{}。
参考文献建议使用~BibTeX~管理,可以生成符合国标~GBT7714~风格的参考文献列表。
模板在~Windows~、~Linux~和MAC OSX下测试通过,更详细的安装步骤请参考第\ref{ch:install}章。


最后,给出一个列表,罗列一下这个模板的功能要点:

\begin{itemize}
\item 使用\XeTeX 引擎处理中文;
\item 包含中文字符的源文件(.tex, .bib, .cfg),编码都使用UTF-8;
\item 使用\BibTeX 管理参考文献。参考文献表现形式(格式)受~.bst~控制,方便在不同风格间切换\cite{Schmidt,Schneider2010,Sha2003,Shan2010,Wang2010},目前生成的列表符合国标GBT7714要求;
\item 可以直接插入EPS/PDF/JPG/PNG格式的图像,并且\emph{不需要}~bounding box~文件(.bb)。
\item 模版采用分章节管理文件,可选择编译,节省时间;论文封面、签名的标题、版权声明等通过WORD转换成pdf,并包含在模版主文件中。

\end{itemize}

\subsection{模板文件布局}
\label{sec:layout}

\begin{lstlisting}[basicstyle=\small\ttfamily,caption={模板文件布局},label=layout,numbers=none]
  ├─ dmuthesis.tex % 论文主源文件
  ├── 使用手册.pdf % 使用手册
  ├── dmuthesis.cls  % 模版类文件
	├── front  % 中英文摘要和创新点摘要
	│   ├── original_abstract.tex
  │   ├── cn_abstract.tex
  │   └── en_abstract.tex
  ├── body  % 各章节源文件
  │   ├── chapter01.tex
  │   ├── chapter02.tex
  │   ├── chapter03.tex
  │   ├── chapter04.tex
  │   └── chapter05.tex
  ├── back  % 篇后部份源文件
  │   ├── appendix.tex
  │   ├── pub.tex
  │   └── ack.tex
  ├── cover  % 封面
  │   └── cover.pdf
  ├── figs % 图片文件
  │   └── chap2
  ├── dmubib.bst % 参考文献格式定义
  ├── reference % 参考文献数据库文件
  │   ├── thesisbib.bib
  │   └── chapter3bib.bib
  ├── dmubib.bst  % 参考文献样式
  ├── clean.bat  % 批处理,清理中间文件
	├── clean.sh  % 批处理,清理中间文件 linux和mac下使用
	├── run.sh %批处理运行, linux和mac下使用
	├── makefile  % 另一专用批处理运行,linux和mac下使用
  └── run.bat % 批处理,WINDOWS用
\end{lstlisting}
